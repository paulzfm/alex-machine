\documentclass[11pt, a4paper]{article}

\usepackage{amsmath}
\usepackage{amssymb}

% fonts
\usepackage{xeCJK}
\setCJKmainfont[BoldFont=SimHei]{SimSun}
\setCJKfamilyfont{hei}{SimHei}
\setCJKfamilyfont{kai}{KaiTi}
\setCJKfamilyfont{fang}{FangSong}
\newcommand{\hei}{\CJKfamily{hei}}
\newcommand{\kai}{\CJKfamily{kai}}
\newcommand{\fang}{\CJKfamily{fang}}

\setmainfont{Times New Roman}

% style
\usepackage[top=2.54cm, bottom=2.54cm, left=3.18cm, right=3.18cm]{geometry}
\usepackage{indentfirst}
\linespread{1.5}
\parindent 2em
\punctstyle{quanjiao}
\renewcommand{\today}{\number\year 年 \number\month 月 \number\day 日}

\usepackage{hyperref}

\renewcommand{\t}{\texttt}

% start of document
\title{\hei Alex 指令集介绍}
\author{\kai \quad 计35 \quad 朱俸民 \quad 2012011894}
\date{\kai \today}

\begin{document}

\maketitle

\section{综述}

Alex 指令集是一种32位定长 RISC 指令集,其设计参考了 MIPS 32、x86 和 v9 三种指令集,与 v9 的系统功能兼容。该指令集结构规整、定义清晰,易于初学者学习和掌握。除了支持基本的32位整数的算术逻辑运算、跳转和访存外,还支持64位浮点操作和基本的系统指令,可以在该指令集上运行 v9 OS、xv6和ucore。完整的指令集共有100条指令,我们选取的一个适合硬件实现的子集仅包含其中的36条指令。

\section{动机}

\subsection{对现有指令集的评价}

\subsubsection{MIPS 32}

MIPS 指令集是目前教学采用最广泛的指令集,其指令精简,尤其是将访存与其他指令分开,易于硬件实现。Alex 指令集的主体参考了 MIPS 的设计,但是由于 MIPS 在过程调用中可能会同时使用寄存器和栈来传参,这一点对于汇编语言的初学者来说容易造成混淆与误解。为了保持过程调用原理的简单性和一致性,Alex 指令集在过程调用上统一使用栈来维护。此外,标准 MIPS 的系统指令比较繁杂,对于实现一个简易的操作系统来说,许多指令我们并不能用上。

\subsubsection{x86}

x86 作为商业上广泛应用的指令集,具有功能完善、兼容性好等诸多优势,但是将其作为教学使用是不太适合的,一方面是因为它是商用的,涉及诸多版权问题;另一方面,变长的 CISC 指令集过于复杂,学生需要花较多时间学习指令的含义。x86 32位的过程调用基于栈,Alex 指令集参考了其过程调用和返回的机制。

\subsubsection{v9}

v9 是专为操作系统教学而采用的指令集,为了让软件层实现简单,v9 指令集的指令条数多,有相当一部分指令实现了很复杂的功能(如内存拷贝、数学函数运算等)。虽然 v9 的目标是教学,但是相当一部分同学反映 v9 指令集并不那么容易理解,主要原因在于 v9 指令集的风格与大家在汇编语言课程上接触的 x86 和 MIPS 相去甚远。同时,v9 指令集的通用寄存器数量很少,这对于汇编程序的编写也是不利的。但是,v9 的系统指令比较精简,支持了操作系统最核心的操作(中断、分页、进程管理),Alex 的系统指令主要参考了 v9 的设计。

\subsubsection{CPU 大实验中的 MIPS 16E}

在“计算机组成原理”的 CPU 大实验中,16位机的标准指令是MIPS 16E的一个子集,该指令集最大的优势在于其硬件实现比较简单。但是,由于指令长度仅有16位,其规整性很差,大多数指令的格式不一致,立即数的长度有5位、8位、11位不等,这无形中加大了正确译码的难度。同时,该指令集缺乏系统指令,不适合在上面跑操作系统。

\subsection{Alex 指令集设计思想}

针对上述介绍的各指令集,我们借鉴其长处,同时回避其缺点,整合设计出了具有如下四个特点的 Alex 指令集。

\subsubsection{规整性}

所有指令定长(32位),且都能通过8位操作码唯一确定,不存在多个功能不同的指令公用操作码的情形。指令均为三个操作数,按照第三个操作数是否为16位立即数分为I型和R型。为了保证立即数都是16位,我们没有引入J型指令。寄存器操作数宽度为4位,可以表示16个通用寄存器。在命名规则上,遵循统一原则,用相同的前缀或者后缀表示相同的功能,如算术指令中用ADD、ADDI和ADDIU分别表示加法、(带符号)立即数加法、无符号立即数加法,访存指令中用 LW、LH、LB、LF 分别表示读取一个字 (word,32位)、一个半字 (half word)、一个字节 (byte)、浮点 (floating-point),系统指令中用 MF (move from) 和 MT (move to) 分别表示从特殊寄存器载入到通用寄存器、从通用寄存器写入特殊寄存器。

\subsubsection{单一性}

每条指令只完成一个逻辑上单一的功能,如某种特定的算术逻辑运算、完成一次跳转、完成一次过程调用、把一个操作数压栈等。访存只能通过 Load/Store 指令,普通指令只能操作寄存器中的数据,这符合 RISC 的思想。在过程调用上,Alex 指令集仅允许通过栈来完成,而不像 MIPS 那样会把寄存器和栈一起使用,这样利于程序员对指令行为的掌控,以便在实现操作系统时更好地操作数据。

\subsubsection{灵活性}

各条指令最多可以有三个操作数,Alex 的惯例是将首个操作数作为目标结果,后面两个操作数作为参数,即与 MIPS 的惯例保持一致。这三个操作数都可以是任意通用寄存器,与 v9 指令集固定寄存器的做法相比,这样可以大大增加指令的通用性,减少不必要的数据中转。即使是在系统指令中,用户依然可以指定用来接收结果或者作为参数的通用寄存器。保持这样的灵活性,也能更好发挥通用寄存器数目多的优势,减少不必要的访存开销。

\subsubsection{易学性}

作为以教学为目标的指令集,最重要的特性就是要简单易学。Alex 指令集充分借鉴了现有的 MIPS、x86 和 v9 指令集,对于汇编语言有所了解的人可以很快掌握该指令集。在设计过程中,我们尽量回避繁琐的设计方法,避免引入让初学者感到迷惑的概念,保持指令集的简单性。例如,MIPS 中的 BLTZAL (Branch on Less Than Zero and Link) 等将条件分支与过程调用结合在一起的指令我们没有采用,而是把分支跳转与过程调用完全分开。由于 Alex 指令集具备真实世界处理器采用的指令集的诸多特征,从 Alex 入门了解汇编语言也是可行的选择。

\section{指令分类}

指令按照功能分为如下8个类别。

\subsection{NOP}

空指令,什么操作也不做。

\subsection{Arithmetic/Logic}

35条,支持整数的加、减、乘、除、取模这些算术操作以及移位、按位与、按位或、按位亦或、按位取反、比较大小这些逻辑操作。

\subsection{Branch/Jump}

9条,完成分支跳转、无条件跳转、过程调用与返回的功能。

\subsection{Load/Store}

11条,完成访存和载入立即数的功能。

\subsection{Stack}

10条,支持栈操作:入栈、出栈。

\subsection{Conversion}

3条,完成整数与浮点数的类型转换。

\subsection{Floating-point}

13条,完成浮点数的算术运算、比较大小和取整。

\subsection{System}

15条,完成输入输出、中断使能、分页使能、设置时钟中断、系统调用和关机等功能。其中 TIME 和 HALT 指令仅在模拟器有用,在硬件设计中没有作用。这些系统指令中仅有用来完成系统调用的 TRAP 指令允许在用户态执行外,其他指令仅能在内核态执行,否则发生异常。

\section{硬件实现参考}

为简化硬件设计实现,在不考虑浮点运算功能与整数乘除法功能的前提下,实现如下23条普通指令

\begin{center}
\begin{tabular}{llllll}
    \t{ADD} & \t{ADDI} & \t{ADDIU} & \t{SUB} & \t{SUBI} & \t{SUBIU} \\
    \t{SHL} & \t{SLR} & \t{SAR} & \t{AND} & \t{OR} & \t{NOT} \\
    \t{XOR} & \t{EQ} & \t{LT} & \t{LTU} & \t{BEQ} & \t{BLT} \\
    \t{JR}  & \t{CALL} & \t{RET} & \t{LW} & \t{SW} & \\
\end{tabular}
\end{center}
和13条系统指令(不包括 TIME 和 HALT),共计36条指令即可完成一个能跑简易操作系统的处理器。如果输入输出通过普通访存指令实现的话,则只需要实现34条指令。

\section{指令系统文档}

参见\url{https://github.com/paulzfm/alex-machine/blob/master/is.md}。

\end{document}
