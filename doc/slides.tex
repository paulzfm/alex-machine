\documentclass{beamer}

% fonts
\usepackage{xeCJK}
\setCJKmainfont[BoldFont=SimHei]{SimHei}
\setCJKfamilyfont{hei}{SimHei}

\renewcommand{\r}{\alert}
\renewcommand{\t}{\texttt}

\renewcommand{\today}{\number\year 年 \number\month 月 \number\day 日}

\usepackage{fontspec}
\setmonofont{Monaco}
\setcounter{secnumdepth}{3}
\setcounter{tocdepth}{3}

\usepackage{hyperref}
\usepackage{xcolor}

\begin{document}

\title{Alex 指令集与模拟器}
\author{朱俸民}
\institute{计35}
\date{\today}

% contents
\AtBeginSection[]
{
    \begin{frame}{目录}
    	\tableofcontents[currentsection]
	\end{frame}
}

\AtBeginSubsection[]
{
    \begin{frame}{目录}
    	\tableofcontents[currentsubsection]
	\end{frame}
}

% titlepage
\begin{frame}
    \titlepage
\end{frame}

\begin{frame}{提纲}
    \tableofcontents
\end{frame}

\section{指令集}

\subsection{综述}

\begin{frame}{综述}
    \begin{itemize}
        \item 32位定长
        \item RISC
        \item 参考了 MIPS 32、x86、v9
        \item 易于初学者学习和掌握
        \item 64位浮点操作
        \item 配套工具链支持
    \end{itemize}
\end{frame}

\subsection{动机}

\begin{frame}{现有指令集功能不足}
    \begin{itemize}
        \item MIPS 32
        \item x86
        \item v9
        \item CPU 大实验中的 MIPS 16E
    \end{itemize}
\end{frame}

\begin{frame}{Alex 特点}
    \begin{itemize}
        \item 规整性
        \item 单一性
        \item 灵活性
        \item 易学性
    \end{itemize}
\end{frame}

\subsection{分类}

\begin{frame}{指令分类}
    \begin{itemize}
        \item NOP (1)
        \item Arithmetic/Logic (35)
        \item Branch/Jump (9)
        \item Load/Store (11)
        \item Stack (10)
        \item Conversion (3)
        \item Floating-point (13)
        \item System (15 - 2)
    \end{itemize}
\end{frame}

\subsection{硬件实现参考}

\begin{frame}{普通指令 (23)}
    \begin{center}
    \begin{tabular}{llllll}
        \t{ADD} & \t{ADDI} & \t{ADDIU} & \t{SUB} & \t{SUBI} & \t{SUBIU} \\
        \t{SHL} & \t{SLR} & \t{SAR} & \t{AND} & \t{OR} & \t{NOT} \\
        \t{XOR} & \t{EQ} & \t{LT} & \t{LTU} & \t{BEQ} & \t{BLT} \\
        \t{JR}  & \t{CALL} & \t{RET} & \t{LW} & \t{SW} & \\
    \end{tabular}
    \end{center}
\end{frame}

\begin{frame}{系统指令 (13)}
    \begin{center}
    \begin{tabular}{llllll}
        \t{ADD} & \t{ADDI} & \t{ADDIU} & \t{SUB} & \t{SUBI} & \t{SUBIU} \\
        \t{SHL} & \t{SLR} & \t{SAR} & \t{AND} & \t{OR} & \t{NOT} \\
        \t{XOR} & \t{EQ} & \t{LT} & \t{LTU} & \t{BEQ} & \t{BLT} \\
        \t{JR}  & \t{CALL} & \t{RET} & \t{LW} & \t{SW} & \\
    \end{tabular}
    \end{center}
\end{frame}

\begin{frame}{指令系统文档}
    \begin{itemize}
        \item \url{https://github.com/paulzfm/alex-machine/blob/master/is.md}
    \end{itemize}
\end{frame}

\section{模拟器}

\begin{frame}{v1}
    \begin{itemize}
        \item \url{https://github.com/paulzfm/alex-machine}
        \item Node.js
        \item 可运行除系统指令外的所有指令
    \end{itemize}
\end{frame}

\begin{frame}{v2}
    \begin{itemize}
        \item \url{https://github.com/paulzfm/v9.js}
        \item 基于已有 v9.js 框架
        \item 前端添加\t{alex.js},移植全部 Alex 指令
    \end{itemize}
\end{frame}

\begin{frame}{仓库}
    \begin{itemize}
        \item \url{https://github.com/paulzfm/alex-machine}
        \item \url{https://github.com/paulzfm/v9.js}
        \item \url{https://github.com/a1exwang/llvm}
        \item \url{https://github.com/a1exwang/lld}
        \item \url{https://github.com/a1exwang/alex-cpu-test}
    \end{itemize}
\end{frame}

\begin{frame}{文档}
    \begin{itemize}
        \item \url{https://github.com/paulzfm/alex-machine/blob/master/doc/doc.pdf}
        \item \url{https://github.com/paulzfm/alex-machine/blob/master/is.md}
    \end{itemize}
\end{frame}

\begin{frame}
    \begin{center}
        {\huge \color{blue}
            Thank you!
        }
    \end{center}

    \begin{center}
        {\huge \color{blue}
            Any questions?
        }
    \end{center}
\end{frame}

\end{document}
